% !TeX root = LAWRGe2023Notes.tex

Throughout the week we will use the language of TQFTs to relate physics and math. The goal of this talk is to introduce the relevant terminology and some definitions.

\subsection{TQFTs}

\subsection{Definition}

We begin by introducing the notion of a $d$-dimensional TQFT.

\begin{defn}
	Let $M,N$ be closed oriented $(d-1)$-manifolds. A \defterm{$d$-dimensional cobordism $W$ from $M$ to $N$} is an oriented $d$-dimensional manifold $W$ together with an identification $\partial W\cong \overline{M}\coprod N$.
\end{defn}

\begin{remark}
	There is also a notion of an unoriented cobordism between two unoriented manifolds, framed cobordism between two framed manifolds and, more generally, a cobordism equipped with a tangential structure.
\end{remark}

Cobordisms define a symmetric monoidal category $\Cob^{or}_{d, d-1}$ as follows:
\begin{itemize}
	\item Its objects are closed oriented $(d-1)$-manifolds.
	\item Morphisms from $M$ to $N$ are diffeomorphism classes of oriented cobordisms from $M$ to $N$.
	\item Composition of a cobordism $W_1$ from $M$ to $N$ and a cobordism $W_2$ from $N$ to $O$ is given by the cobordism $W_1\coprod_N W_2$ from $M$ to $O$.
	\item The symmetric monoidal structure is given by disjoint union of manifolds.
\end{itemize}

\begin{defn}
	An \defterm{oriented $d$-dimensional TQFT} is a symmetric monoidal functor $Z\colon \Cob_{d, d-1}\rightarrow \Vect$ to the category of ($\C$-)vector spaces with tensor product as the monoidal structure.
\end{defn}

The physical idea of the definition is as follows:
\begin{itemize}
	\item For a closed $(d-1)$-manifold $M$ we have a vector space $Z(M)$. It is the vector space of states of the TQFT (often a Hilbert space in physical examples).
	
	\item For a closed $d$-manifold $W$ we have a number $Z(W)$. It is the partition function of the TQFT on $W$.
	
	\item For a cobordism $W$ from $M$ to $N$ we get a linear map $Z(W)\colon Z(M)\rightarrow Z(N)$. It is the \emph{transition amplitude} ($S$-matrix) associated to the cobordism $W$.
\end{itemize}

\subsection{Extending down}

Given a decomposition $W = W_1\coprod_M W_2$ of a closed oriented $d$-manifold into a union of two manifolds along their common boundary, one can compute the partition function as
\[Z(W) = Z(W_2)(Z(W_1)(1)),\]
where
\[Z(W_1)\colon \C\longrightarrow Z(M),\qquad Z(W_2)\colon Z(M)\longrightarrow \C.\]

This allows one to compute the partition function by decomposing a manifold into pieces. This is related to the principle of locality of a QFT. Full locality will also allow us to compute the partition function by decomposing the boundary $M$ into pieces. This can be made precise by extending the category $\Cob^{or}_{d, d-1}$ to a 2-category or even a higher category as follows. Let $\Cob^{or}_{d, d-1, d-2}$ be the symmetric monoidal 2-category as follows:
\begin{itemize}
	\item Its objects are closed oriented $(d-2)$-manifolds.
	\item 1-morphisms from $M$ to $N$ are oriented $(d-1)$-dimensional cobordisms $W$ from $M$ to $N$.
	\item 2-morphisms from $W_1\colon M\rightarrow N$ to $W_2\colon M\rightarrow N$ are diffeomorphism classes of $d$-dimensional cobordisms between $W_1$ and $W_2$.
\end{itemize}

One can also extend it all the way down and define the symmetric monoidal $d$-category $\Cob^{or}_d$ whose objects are closed oriented 0-manifolds (disjoint unions of oriented points), 1-morphisms are 1-dimensional cobordisms and so on.

To define TQFTs we also need to extend the target category $\Vect$ down. For instance, for once-extended TQFTs we are looking for a symmetric monoidal bicategory $\cC$ (usually it is the bicategory of some class of categories) with the property that $\Hom_\cC(1, 1)\cong \Vect$. Similarly, for fully extended TQFTs we are looking for a symmetric monoidal $d$-category $\cC$ with a similar property for top-level morphisms.

\begin{defn}
	Let $\cC$ be a (linear) symmetric monoidal $d$-category. A \defterm{fully extended TQFT} is a symmetric monoidal functor $Z\colon \Cob^{or}_d\rightarrow \cC$.
\end{defn}

Note that given any fully extended TQFT we obtain higher-categorical structures irrespectively of the target $\cC$:
\begin{itemize}
	\item If $M$ is a closed oriented $(d-1)$-manifold, $Z(M)$ is a vector space. We can think of $Z(M)$ as an element of the vector space $\Hom_\cC(Z(\varnothing^{d-1}), Z(M))$.
	
	\item If $M$ is a closed oriented $(d-2)$-manifold, $\Hom_\cC(Z(\varnothing^{d-2}), Z(M))$ is a category. In fact, the structure of an oriented TQFT will induce a Calabi--Yau structure on this.
	\item \dots
\end{itemize}

\subsection{Extending up}

Let $M$ be a closed oriented $d$-manifold and $\Diff(M)$ the topological group of orientation-preserving diffeomorphisms of $M$. There is a natural map
\[\MCG(M) = \pi_0\Diff(M)\longrightarrow \Aut_{\Bord^{or}_{d, d-1}}(M)\]
given by considering $W=M\times [0, 1]$ with the identification $\partial W\cong \overline{M}\coprod M$ twisted by a diffeomorphism. The reason that isotopic diffeomorphisms give rise to the same morphisms is that in the definition of $\Bord^{or}_{d, d-1}$ we identify diffeomorphic cobordisms.

The full homotopy type of the diffeomorphism group can be encoded if we work in the framework of $\infty$-categories. Namely, there is a symmetric monoidal $\infty$-category $\Bord^{or}_{d, d-1}$ which has the following informal description:
\begin{itemize}
	\item Its objects are closed oriented $(d-1)$-manifolds.
	\item 1-morphisms from $M$ to $N$ are oriented cobordisms from $M$ to $N$.
	\item 2-morphisms are given by diffeomorphisms of cobordisms.
	\item 3-morphisms are given by isotopies of diffeomorphisms.
	\item ...
\end{itemize}

Similarly, there is a symmetric monoidal $(\infty, d)$-category $\Bord^{or}_d$.

\begin{defn}
	Let $\cC$ be a symmetric monoidal $(\infty, d)$-category. A \defterm{fully extended TQFT} is a symmetric monoidal functor $Z\colon \Bord^{or}_d\rightarrow \cC$.
\end{defn}

In physics the state space on a closed oriented $(d-1)$-manifold $M$ is often a chain complex $Z(M)\in\Ch$ (with the differential the BRST differential coming from gauge symmetries and/or supersymmetric twisting). So, while on the level of cohomology there is an action of the mapping class group $\MCG(M)$ on $\rH^\bullet(Z(M))$, on the chain level it should extend to a homotopy-coherent action of $\rC_\bullet(\Diff(M))$ (equipped with the Pontryagin product) on the chain complex $Z(M)$. We will encounter the following two versions of this action:
\begin{itemize}
	\item Consider a $d$-dimensional TQFT $Z$ (for $d\geq 2$) and the chain complex $Z(S^{d-1})$. The natural $S^1$-action on $S^{d-1}$ induces a $\C_\bullet(S^1)$-action on the chain complex $Z(S^{d-1})$. This action boils down to a square-zero degree $-1$ operation $B\colon Z(S^{d-1})\rightarrow Z(S^{d-1})$.
	
	\item Consider a $d$-dimensional TQFT $Z$ (for $d\geq 3$) and the category $Z(S^{d-2})$. The natural $S^1$-action on $S^{d-2}$ induces a natural automorphism of the identity functor on $Z(S^{d-2})$.
\end{itemize}

If $M$ is a closed oriented $d$-manifold, the partition function $Z(M)$ is merely a number. So, the higher-categorical structure is irrelevant in this case and we simply have that $Z(M)$ is invariant under $\Diff(M)$. It turns out to be useful to phrase this condition by saying that
\[Z(M)\in\rH^0(\B\Diff(M); \C).\]

\subsection{Boundary conditions}

The notion of a relative TQFT was introduced by Freed--Teleman and Johnson-Freyd--Scheimbauer. We will not give a precise definition, but will just indicate the main idea.

Suppose $Z\colon \Cob^{or}_{d, d-1}\rightarrow \Vect$ be a $d$-dimensional TQFT. The partition function $Z(M)$ \emph{as a number} makes sense only for a closed oriented $d$-manifold. Given a \emph{boundary condition}, we can evaluate the theory on manifolds with boundary as follows:
\begin{itemize}
	\item For any closed oriented $(d-1)$-manifold $M$ we have the space of states $Z(M)$.
	\item The boundary condition defines a distinguished vector $Z^\partial(M)\in Z(M)$.
	\item Given a compact oriented $d$-manifold $W$ with boundary $M$ we may view it as a cobordism $M\rightarrow \varnothing$. In particular,
	\[Z(W)\colon Z(M)\longrightarrow \C.\]
	So, the partition function of the TQFT on $W$ with the given boundary condition is
	\[Z(W)(Z^\partial(M)).\]
\end{itemize}

We can also talk about boundary conditions to once-extended or fully extended TQFTs. Then:
\begin{itemize}
	\item For any closed oriented $(d-k)$-manifold $M$ we have a $(k-1)$-category $\Hom_\cC(Z(\varnothing^{d-k}), Z(M))$.
	\item The boundary condition defines a distinguished object $Z^\partial(M)\in \Hom_\cC(Z(\varnothing^{d-k}), Z(M))$.
\end{itemize}

For instance, on the level of the point we get a distinguished object $Z^\partial(\pt)\in\Hom_\cC(1, Z(\pt))$ in the $(d-1)$-category $\Hom_\cC(1, Z(\pt))$. So, we can think of this as the $(d-1)$-category of \emph{boundary conditions} (more precisely, fully local boundary conditions correspond to suitably dualizable objects of this $(d-1)$-category).

\subsection{2d mirror symmetry}

I will end this lecture by explaining the TQFT ideas behind the usual two-dimensional mirror symmetry as a warm up for three-dimensional mirror symmetry. We have the following 2d TQFTs:
\begin{itemize}
	\item Let $M$ be a symplectic manifold. Then one can define the 2d A-model $Z_{2dA, M}$. The category of boundary conditions $Z_{2dA, M}(\pt)$ is some version of the Fukaya category of $M$.
	
	\item $M$ be a smooth complex algebraic variety. Then one can define the 2d B-model $Z_{2dB, M}$. The category of boundary conditions $Z_{2dB, M}(\pt)$ is some version of the derived category of coherent sheaves on $M$.
\end{itemize}

There are also equivariant versions of these 2d TQFTs:
\begin{itemize}
	\item Given a (real) Lie group $G$ acting in a Hamiltonian way on a symplectic manifold $M$ there is an equivariant 2d A-model.
	\item Given a complex algebraic group $G_\C$ acting on a smooth complex algebraic variety $M$ there is an equivariant 2d B-model.
\end{itemize}

\begin{remark}
	If $M$ is not compact, these TQFTs are not defined on all 2-dimensional cobordisms.
\end{remark}

\begin{remark}
	Even though the framed TQFTs are well-defined, there is an ``orientation anomaly'' which complicates the definition of the oriented TQFT. The partition function on a surface $\Sigma_g$ of genus $g$ defines an element of $\rH^{2(g-1)\dim_\R M}(\B\Diff(\Sigma_g); \C)$ rather than an element of $\rH^0(\B\Diff(\Sigma_g); \C)$. For instance, the underlying number is zero for $g\neq 1$.
\end{remark}

The statement of 2-dimensional homological mirror symmetry can be formulated as follows. We say a symplectic manifold $M$ is 2d mirror to a complex algebraic variety $M^\vee$ if
\[Z_{2dA, M}\cong Z_{2dB, M^\vee}.\]

This contains the following statements:
\begin{itemize}
	\item An equivalence of Calabi-Yau categories.
	\[Z_{2dA, M}(\pt)\cong Z_{2dB, M^\vee}(\pt).\]
	In practice the left-hand side is a version of the Fukaya category of $M$ and the right-hand side is a version of the derived category of coherent sheaves on $M^\vee$.
	\item An equivalence of commutative algebras
	\[\rH^\bullet(Z_{2dA, M}(S^1))\cong \rH^\bullet(Z_{2dB, M^\vee}(S^1)).\]
	In fact, there is a Gerstenhaber structure (explained in the next lecture) on both sides which is also preserved.
\end{itemize}