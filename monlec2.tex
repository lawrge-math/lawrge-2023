% !TeX root = LAWRGe2023Notes.tex

In this lecture $Z$ denotes some $d$-dimensional TQFT. For simplicity I will assume that it is fully extended, but many statements make sense with partially extended TQFTs. I will assume that the TQFT is valued in the ($\infty$-)category of chain complexes.

\subsection{Local and line operators}

Besides computing partition functions, in physics one is often interested in computing correlation functions of some local operators. Let us introduce them using the following heuristic idea.

Suppose $M$ is a closed oriented $d$-manifold and $x\in M$ is a point with an insertion of a ``local operator'' $\cO$. By locality one should be able to compute the partition function as follows:
\begin{itemize}
\item Consider a ball $D\subset M$ around $x$ and let $S^{d-1}\subset M$ be its boundary.
\item $Z(M\setminus D)$ defines a map $Z(S^{d-1})\rightarrow \C$. The local operator defines a map $Z(D_{\cO})\colon \C\rightarrow Z(S^{d-1})$ and the partition function on $M$ is the composite of these two maps.
\end{itemize}

If we are being agnostic about local operators, we may observe that the only thing we have used about them is the vector of $Z(S^{d-1})$ that they define. This leads us to the following definition.

\begin{defn}
Let $Z$ be a $d$-dimensional TQFT. The \defterm{space of local operators} is the chain complex $Z(S^{d-1})$.
\end{defn}

One also considers defects given by extended objects: lines, surfaces, ... embedded in $M$. Besides local operators, we will only encounter line operators this week. We can think of them as follows:
\begin{itemize}
\item A line operator is specified by a defect supported on a knot $K\subset M$. The same analysis as before shows that we can compute the partition function if we know the corresponding vector in $Z(S^{d-2}\times K)$.

\item One often only considers ``local'' line operators which themselves obey cutting and gluing axioms of a TQFT. These local line operators define an object of the category $Z(S^{d-2})$.
\end{itemize}

This motivates the following definition.

\begin{defn}
Let $Z$ be a $d$-dimensional TQFT.
\begin{itemize}
\item The \defterm{space of line operators} is $Z(S^{d-2}\times S^1)$.
\item The \defterm{category of line operators} is $Z(S^{d-2})$.
\end{itemize}
\end{defn}

\subsection{$\bE_d$-algebras}

Our next goal is to explain algebraic structures present on the space of local and line operators. Given any cobordism $W$ from $k$ copies of $S^{d-1}$ to $S^{d-1}$ we get an algebraic operation
\[Z(W)\colon Z(S^{d-1})^{\otimes k}\longrightarrow Z(S^{d-1})\]
on the space of local operators in any TQFT. We will now investigate operations coming from cobordisms ``with no topology''.

\begin{defn}
Fix a dimension $d$.
\begin{itemize}
\item \[\bE_d(k) = \Emb^{fr}(D^{\coprod k}, D)\]
is the space of (smooth) framed embeddings of $k$ $d$-dimensional open disks $D$ into a given disk $D$.

\item \[\bE^{fr}_d(k) = \Emb(D^{\coprod k}, D)\]
is the space of (smooth) oriented embeddings of $k$ $d$-dimensional open disks $D$ into a given disk $D$.
\end{itemize}
\end{defn}

There are natural composition maps which make $\bE_d$ and $\bE^{fr}_d$ into operads. In particular, we can talk about their algebras.

\begin{example}
The operads $\bE_1$ and $\bE^{fr}_1$ are both equivalent to the associative operad.
\end{example}

\begin{example}
There is a natural action of $\SO(d)$ on $\bE_d$, so that a $\bE^{fr}_d$-algebra is an $\bE_d$-algebra equipped with a compatible $\SO(d)$-action.
\end{example}

Given an embedding $D^{\coprod k}\hookrightarrow D$ we obtain a cobordism from $(S^{d-1})^{\coprod k}$ to $S^{d-1}$ by removing the interiors of the embedded disks. In particular, we obtain a natural map
\[\rC_\bullet(\bE^{fr}_d(k); \C)\otimes_\C Z(S^{d-1})^{\otimes k}\longrightarrow Z(S^{d-1})\]
for any oriented TQFT. Similarly, if $Z$ is a framed TQFT we get a natural map
\[\rC_\bullet(\bE_d(k); \C)\otimes_\C Z(S^{d-1})^{\otimes k}\longrightarrow Z(S^{d-1}).\]
Both maps are compatible with compositions and we obtain the following result:
\begin{itemize}
\item If $Z$ is a framed TQFT, the chain complex of local operators $Z(S^{d-1})$ is an $\bE_d$-algebra.
\item If $Z$ is an oriented TQFT, the chain complex of local operators $Z(S^{d-1})$ is a framed $\bE_d$-algebra.
\end{itemize}

Up to homotopy the spaces of embeddings may be identified as follows.

\begin{prop}
There are homotopy equivalences
\[\bE_d(k)\cong \Conf_k(\R^d),\qquad \bE^{fr}_d(k)\cong \SO(d)^k\times \Conf_k(\R^d),\]
where $\Conf_k(\R^d)$ is the configuration space of $k$ distinct ordered points in $\R^d$.
\end{prop}

Using the above description one can show the following:
\begin{itemize}
\item An $\bE_2$-algebra in categories is a braided monoidal category.
\item An $\bE^{fr}_2$-algebra in categories is a balanced monoidal category, i.e. there is an extra automorphism of the identity functor, the \emph{balancing} $\theta$, which satisfies
\[\theta_{x\otimes y} = \sigma_{y, x}\circ \sigma_{x, y}\circ (\theta_x\otimes \theta_y).\]
\end{itemize}

So, the category of line operators in a 3-dimensional TQFT is a balanced monoidal category.

\subsection{$\bP_d$-algebras}

To describe $\bE_d$-algebras in chain complexes, let us first introduce a related notion.

\begin{defn}
A \defterm{$\bP_d$-algebra} is a commutative dg algebra $A$ equipped with a bracket of cohomological degree $1-d$ (inducing a Lie structure on $A[d-1]$) satisfying the Leibniz rule
\[\{a, bc\} = \{a, b\}c + (-1)^{|b||c|}\{a, c\}b\]
for $a,b,c\in A$.
\end{defn}

\begin{remark}
A $\bP_2$-algebra is known as a \emph{Gerstenhaber algebra}.
\end{remark}

Let $\bP_d(k)$ be the vector space of all operations $A^{\otimes k}\rightarrow A$ on a $\bP_d$-algebra. We can formalize it as follows: define $\bP_d(k)$ to be the subspace of the free $\bP_d$-algebra on degree 0 variables $x_1, \dots, x_k$ consisting of expressions where each $x_i$ appears exactly once. For instance:
\begin{itemize}
\item $\bP_d(1)\cong \C$ spanned by the identity map $A\rightarrow A$.
\item $\bP_d(2)\cong \C\oplus \C[d-1]$ spanned by the commutative multiplication $m\colon A\otimes A\rightarrow A$ and $\{-, -\}\colon A\otimes A\rightarrow A[1-d]$ by the Poisson bracket.
\end{itemize}

We have the following claim.

\begin{defn}
Suppose $d\geq 2$. Then there is an isomorphism of graded vector spaces $\rH_\bullet(\bE_d(k); \C)\cong \bP_d(k)$.
\end{defn}

\begin{remark}
In fact, both $\bE_d$ and $\bP_d$ are operads and there is an equivalence $\rC_\bullet(\bE_d; \C)\cong \bP_d$ of graded linear operads.
\end{remark}

As a corollary, given an $\bE_d$-algebra $A$, its homology $\rH_\bullet(A)$ has a natural structure of a $\bP_d$-algebra.

\begin{example}
Let $Z$ be a 3d TQFT. Then the cohomology of the space of local operators $\rH^\bullet(Z(S^2))$ carries a graded commutative multiplication as well as Poisson bracket of degree $-2$.
\end{example}

\subsection{$\Omega$-deformation}

We will now explain an important construction with $\bE_d$-operads which is known in physics as the procedure of $\Omega$-deformation.

Consider the $\bE_d$-operad equipped with its natural $\SO(d)$-action. There is a natural inclusion of operads $\bE_{d-2}\hookrightarrow \bE_d$ which is $\SO(d-2)\times \SO(2)$-equivariant, where the $\SO(2)$-action on the left is trivial.

\begin{thm}
The inclusion of operads $\bE_{d-2}\hookrightarrow \bE_d$ realizes $\bE_{d-2}$ as the space of fixed points of the $\SO(2)$-action on $\bE_d$.
\label{thm:Enfixedpoints}
\end{thm}

To state an important corollary, let us first recall a few basics of equivariant localization. Given a space $X$ with an action of a topological group $G$ we may consider equivariant homology $\rH^G_\bullet(X)$ and cohomology $\rH^\bullet_G(X)$ which are both modules over $\rH^\bullet_G(\pt)=\rH^\bullet(\B G)$.

\begin{example}
We have $\B\SO(2)=\CP^\infty$, so $\rH^\bullet(\B\SO(2))=\C[\epsilon]$, where $\deg(\epsilon)=2$.
\end{example}

\begin{thm}[Equivariant localization]
Let $G$ be a topological group. Let $Y$ be a space with a $G$-action. Let $X$ be a topological space equipped with a trivial $G$-action and a $G$-equivariant map $X\rightarrow Y$ which realizes $X$ as the space of fixed points of the $G$-action on $Y$. Then the induced map
\[\rH^G_\bullet(X)\otimes_{\C[\epsilon]} \C(\epsilon) \longrightarrow \rH^G_\bullet(Y)\otimes_{\C[\epsilon]} \C(\epsilon)\]
is an isomorphism.
\end{thm}

Combining the theory of equivariant localization and \ref{thm:Enfixedpoints} we get the following.

\begin{thm}
Let $A$ be a framed $\bE_d$-algebra. Consider the induced $\SO(2)\subset\SO(d)$-action on $A$. Then $A^{\SO(2)}\otimes_{\C[\epsilon]} \C(\epsilon)$ is a framed $\bE_{d-2}$-algebra.
\end{thm}

\begin{example}
Let $Z$ be a 3d TQFT and consider the framed $\bE_3$-algebra structure on the space of local operators $Z(S^2)$. Recall that its cohomology $\rH^\bullet(Z(S^2))$ carries a natural (graded) Poisson structure. The equivariant localization
\[\rH^\bullet_{\SO(2)}(Z(S^2))\otimes_{\C[\epsilon]} \C(\epsilon)\]
is an associative algebra which provides a deformation quantization of $\rH^\bullet(Z(S^2))$ (with the quantization parameter being $\epsilon$).
\end{example}

\subsection{Swiss-cheese algebras}

I will end this lecture by describing algebraic structures appearing in TQFTs with boundary conditions.

Let $Z$ be a $d$-dimensional TQFT with a chosen boundary condition $Z^\partial$. We can extract the following kinds of algebras:
\begin{itemize}
\item The space of \emph{bulk local operators} in $Z$, i.e.
\[A = Z(S^{d-1}),\]
carries the structure of a framed $\bE_d$-algebra.

\item The space of \emph{boundary local operators}, i.e.
\[B = Z(D^{d-1})(Z^\partial(S^{d-2})),\]
carries the structure of a framed $\bE_{d-1}$-algebra. Namely, let $H$ be the $d$-dimensional half-ball. Consider the space
\[\Emb^\partial(H^{\coprod k}, H)\]
of oriented embeddings of $k$ $d$-dimensional half-balls into a single one so that the boundaries are embedded into the boundaries. Retracting the half-ball to its boundary (a $(d-1)$-dimensional ball) identifies this space with $\bE^{fr}_{d-1}$.

\item In addition, there is an action of $A$ on $B$ as follows. Let $D$ be a $d$-dimensional ball and $H$ a $d$-dimensional half-ball. Any embedding $D^{\coprod l}\coprod H^{\coprod k}\hookrightarrow H$ (so that the boundaries of the half-balls are embedded in the boundary of the bigger half-ball) gives rise to an operation
\[A^{\otimes l}\otimes B^{\otimes k}\longrightarrow B.\]
\end{itemize}

The pair $(A, B)$ together with the operations described above is known as a $d$-dimensional \defterm{Swiss-cheese algebra}. We will encounter the following manifestation of this structure.

\begin{example}
Let $Z$ be a 3d TQFT with a boundary condition $Z^\partial$. Let $A$ be the $\bE_3$-algebra of bulk local operators and $B$ the $\bE_2$-algebra of boundary local operators. $A$ carries a degree $-2$ Poisson structure. There is a map of graded commutative algebras $A\rightarrow B$ and the induced map $\Spec B\rightarrow \Spec A$ is coisotropic.
\end{example}

\begin{remark}
There is a homotopy notion of a coisotropic submanifold which is precisely defined in terms of a Swiss-cheese algebra structure.
\end{remark}


Recall that $\bE_n(k)$ is homotopy equivalent to $\Conf_k(\R^n)$, the configuration space of $k$ distinct points in $\R^n$.

Recall that a $\bP_n$-algebra is a dg commutative algebra $A$ equipped with a bracket $\{-, -\}$ of cohomological degree $1-n$ (inducing a Lie structure on $A[n-1]$) satisfying the relation $\{a, bc\} = \{a, b\}c + (-1)^{|b||c|}\{a, c\}b$. Let $\bP_n(k)$ be the subspace of the free $\bP_n$-algebra on degree $0$ variables $x_1, \dots, x_k$ consisting of expressions where each $x_i$ appears exactly once. For instance, $\{\{x_1, x_2\}, \{x_3, x_4\}\}$ is an element of $\bP_n(4)$ of cohomological degree $3(1-n)$.

\subsection{Exercises}
\begin{exercise}
Consider the map $\Conf_k(\R^n)\rightarrow \Conf_{k-1}(\R^n)$ given by forgetting the last point. Show that its fiber $F_k$ is homotopy equivalent to a wedge of $(k-1)$ spheres $S^{n-1}$.
\end{exercise}

\begin{exercise}
The Leray--Serre spectral sequence for the fibration $F_k\hookrightarrow \Conf_k(\R^n)\rightarrow \Conf_{k-1}(\R^n)$ degenerates (using the Leray--Hirsch theorem), so that one may identify
\[\rH_\bullet(\Conf_k(\R^n);\Q)\cong \rH_\bullet(\Conf_{k-1}(\R^n);\Q)\otimes \rH_\bullet(F_k;\Q).\]
Find $\rH_\bullet(\bE_n(k);\Q)$ for $k=1,2,3$.
\end{exercise}

\begin{exercise}
Describe the graded vector space $\bP_n(k)$ for $k=1,2,3$ and find an isomorphism
\[\rH_\bullet(\bE_n(k);\Q)\cong \bP_n(k)\]
for $n\geq 2$.
\end{exercise}

\begin{exercise}
(*) Consider the $S_2$-action on $\bE_2(2)\sim S^1$ given by reflection around the origin. Let $\cC$ be a category. Show that an $S_2$-equivariant map
\[S^1\times \cC^{\times 2}\longrightarrow \cC\]
is the same as a pair $(\otimes, \sigma)$ consisting of a functor $\otimes\colon \cC\times \cC\rightarrow \cC$ as well as a natural isomorphism
\[\sigma_{x, y}\colon x\otimes y\xrightarrow{\sim} y\otimes x.\]
\end{exercise}

